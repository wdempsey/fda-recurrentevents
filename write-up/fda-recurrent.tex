\documentclass[11pt]{amsart}
\usepackage{amsmath}
\usepackage{tikz}
\usetikzlibrary{matrix,arrows}
\usepackage{amsfonts,amssymb,amsthm, txfonts, pxfonts,amscd}
\usepackage[stretch=10]{microtype}
%\usepackage{hyperref}
\def\struckint{\mathop{%
\def\mathpalette##1##2{\mathchoice{##1\displaystyle##2}%
 {##1\textstyle##2}{##1\scriptstyle##2}{##1\scriptscriptstyle##2}}%
\mathpalette
{\vbox\bgroup\baselineskip0pt\lineskiplimit-1000pt\lineskip-1000pt
\halign\bgroup\hfill$}
{##$\hfill\cr{\intop}\cr\diagup\cr\egroup\egroup}%
}\limits}
\usepackage{natbib}
\usepackage{color}
\usepackage{booktabs,caption,fixltx2e}
\usepackage[flushleft]{threeparttable}

\usepackage{amsmath}
\usepackage{tikz}
\usetikzlibrary{matrix,arrows}
\usepackage{amsfonts,amssymb,amsthm, txfonts, pxfonts,amscd}
%\usepackage[stretch=10]{microtype}
%\usepackage{hyperref}
%\def\struckint{\mathop{%
%\def\mathpalette##1##2{\mathchoice{##1\displaystyle##2}%
% {##1\textstyle##2}{##1\scriptstyle##2}{##1\scriptscriptstyle##2}}%
%\mathpalette
%{\vbox\bgroup\baselineskip0pt\lineskiplimit-1000pt\lineskip-1000pt
%\halign\bgroup\hfill$}
%{##$\hfill\cr{\intop}\cr\diagup\cr\egroup\egroup}%
%}\limits}

\newcommand{\sam}[1]{{\color{blue}{#1}}}
\newcommand{\walt}[1]{\textcolor{red}{[WD:\ #1]}}

\setlength{\textwidth}{6.5in}
\setlength{\textheight}{8.5in}
\setlength{\topmargin}{-0.25in}
\setlength{\evensidemargin}{0in}
\setlength{\oddsidemargin}{0in}
%\setlength{\evensidemargin}{.25in}
%\setlength{\oddsidemargin}{.25in}


\usepackage{setspace}
\doublespacing

\usepackage{soul,color, mathtools}

\def\pr{\mathop{\text{pr}}\nolimits}

\def\Bka{{\it Biometrika}}
\def\AIC{\textsc{aic}}
\def\T{{ \mathrm{\scriptscriptstyle T} }}
\def\v{{\varepsilon}}
\def\partitionsn{\mathop{\mathcal{P}_{[n]}}\nolimits}
\def\partitionsN{\mathop{\mathcal{P}_{\infty}}\nolimits}
\def\Dmn{\mathop{{D}_{m,n}}\nolimits}
\def\symmetricn{\mathop{\mathcal{S}_{n}}\nolimits}
\def\PE{\mathop{\rm Pitman\mbox{-}Ewens}\nolimits}
\def\per{\mathop{\rm per}\nolimits}
\def\U{\mathop{\mathcal{U}_{}}\nolimits}
\def\Nb{\mathop{\mathbb{N}_{}}\nolimits}
\def\Nbb{\mathop{\mathbf{N}_{}}\nolimits}
\def\nbb{\mathop{\mathbf{n}_{}}\nolimits}
\def\Xbb{\mathop{\mathbf{X}_{}}\nolimits}
\def\fin{\mathop{\text{fin}}\nolimits}
\def\pr{\mathop{\text{pr}}\nolimits}
\def\E{\mathcal{E}}
\def\G{\mathcal{G}}
\def\deg{\text{deg}_{}}
\def\equalinlaw{=_{\mathcal{D}}}
\def\Fk{\mathop{\mathcal{F}_k^{\downarrow}}\nolimits}
\def\F{\mathop{\mathcal{F}^{\downarrow}}\nolimits}
\def\size{\mathop{\text{size}}\nolimits}
\def\Ycong{\mathop{\EY}\nolimits}
\def\EIsig{\mathop{\mathcal{E}_{\mathcal{I}}^{\sigma}}\nolimits}
\def\EY{\mathop{\mathcal{E}_{Y}}\nolimits}
\def\finp{\mathop{\fin(\mathcal{P})}\nolimits}
\def\En{\mathop{\mathfrak{E}_{[n]}}\nolimits}
\def\EN{\mathop{\mathfrak{E}_{\Nb}}\nolimits}
\def\logit{\mathop{\text{logit}_{}}\nolimits}
\def\EI{\mathop{\mathcal{E}_{\mathcal{I}}}\nolimits}
\def\ES{\mathop{\mathfrak{E}_S}\nolimits}
\def\Ifk{\mathop{{I}_{}}\nolimits}
%\def\mathcal{I}{\mathop{\mathcal{I}_{}}\nolimits}
\def\Ycong{\mathop{\EY}\nolimits}
\def\EIsig{\mathop{\mathcal{E}_{\mathcal{I}}^{\sigma}}\nolimits}
\def\EY{\mathop{\mathcal{E}_{Y}}\nolimits}
\def\finp{\mathop{\fin(\mathcal{P})}\nolimits}
\def\En{\mathop{\mathfrak{E}_{[n]}}\nolimits}
\def\EN{\mathop{\mathfrak{E}_{\Nb}}\nolimits}
\def\logit{\mathop{\text{logit}_{}}\nolimits}
\def\EI{\mathop{\mathcal{E}_{\mathcal{I}}}\nolimits}
\def\ES{\mathop{\mathfrak{E}_S}\nolimits}



\DeclareRobustCommand{\citeext}[1]{\citeauthor{#1} (\citeyear{#1})}
\DeclareRobustCommand{\citeint}[1]{(\citeauthor{#1}, \citeyear{#1})}

\newtheorem{thm}{Theorem}[section]
\newtheorem{lemma}[thm]{Lemma}
\newtheorem{prop}[thm]{Proposition}
\newtheorem{cor}[thm]{Corollary}
\newtheorem{defn}[thm]{Definition}
\newtheorem{example}[thm]{Example}
\newtheorem{meas}[thm]{Measurement}
\newtheorem{conj}[thm]{Conjecture}
\newtheorem{assumption}[thm]{Assumption}
\newtheorem{rmk}[thm]{Remark}%\endlocaldefs

% change default numbering for enumerate environment to be in parentheses

\makeatletter

\def\S{\mathcal{S}}
\def\indep{\mathrel{\rlap{$\perp$}\kern1.6pt\mathord{\perp}}}
\def\R{\mathcal{R}}
\def\H{\mathcal{H}}
\def\E{\mathbb{E}}
\def\Y{{\bf Y}}
\def\Cov{\text{Cov}}
\def\one{{\bf 1}}
\def\diag{\text{diag}}
\def\given{\, | \,}
\def\Given{\, \big | \,}
\def\Nat{\mathbb{N}}
\def\Real{\mathbb{R}}
\def\bft{{\bf t}}
\def\bfx{{\bf x}}
\def\bfp{{\bf p}}
\def\bfT{{\bf T}}
\def\dotminussym#1#2{%
  \setbox0=\hbox{$\m@th#1-$}%
  \kern.5\wd0%
  \hbox to 0pt{\hss\hbox{$\m@th#1-$}\hss}%
  \raise.6\ht0\hbox to 0pt{\hss$\m@th#1.$\hss}%
  \kern.5\wd0}
\newcommand{\dotminus}{\mathbin{\mathpalette\dotminussym{}}}
\DeclareMathOperator*{\argmax}{arg\,max}
\DeclareMathOperator*{\argmin}{arg\,min}

\mathchardef\mhyphen="2D

% display breaks


\begin{document}


\title{Recurrent event analysis in the presence of functional
  covariates via random subsampling}
\author{Walter Dempsey}
\address {Department of Statistics, 
  Harvard University, One Oxford Street
   Cambridge, MA  02138, USA}
 \email{wdempsey@fas.harvard.edu}

\date{\today}

\begin{abstract}
Digital monitoring studies collect real-time high frequency data via
mobile sensors in the subjects' natural environment.  
This data can be used to model the impact of changes in physiology on
recurrent event outcomes such as smoking, drug use, alcohol use, or
self-identified moments of suicide ideation. 
Likelihood calculations for the recurrent event analysis, however,
become computationally prohibitive in this setting.  
% Unlike classic joint models, event risk is a function of
% recent sensor data, not simply current sensor values.  
Motivated by this, a random subsampling framework is proposed for
computationally efficient estimation of the effect of time-varying
covariates on the recurrent event process. A subsampling-unbiased
estimator for the cumulative hazard enters into an approximation of
log-likelihood. The proposed estimator then maximizes this
approximation. The estimator has two sources of variation: the first
due to the recurrent event model and the second due to subsampling.
The latter can be reduced by increasing the sampling rate; however,
this leads to increased computational costs. 
Simulations demonstrate the method and efficiency-computation
tradeoff. We then illustrate the approach using data from a digital
monitoring study of suicide ideation.
\end{abstract}

\keywords{recurrent events; probabilistic subsampling; estimating equations;
high frequency time series; suicidal ideation}


\maketitle

\section{Introduction}

Advancement in mobile technology has led to the rapid integration of
mobile and wearable sensors into behavioral health~\citep{Freeetal2013}.
Take HeartSteps, for example, a mobile health (mHealth) study designed
to increase physical activity in sedentary adults
\citep{KlasnjaHS2019}. Here, a Jawbone sensor is used to monitor step
count every minute of the participant's study day.
Of interest in many mHealth studies is the relation of such
real-time high frequency sensor data to an adverse, recurrent event
process. In the smoking cessation mHealth study~\citep{}, for
example, the relation between a time-varying sensor-based measure of
physiological stress and the smoking lapse process is of interest.
In a suicidal ideation mHealth study~\citep{}, the relation of
electrodermal activity (EDA) and accelerometer with self-identified
moments of suicidal ideation is of interest.


The goal of this paper is two-fold: (1) to discuss the appropriate
choice of a statistical model for joint high-frequency sensor data and
the recurrent event data, and (2) to construct a robust,
easy-to-implement method for parameter estimation and inference. 
For (1), we discuss an important issue regarding measurement-error
models when paired with recurrent event outcomes.
For (2), we introduce a random subsampling procedure that has
several benefits.  First, the resulting inference is unbiased;
however, there is a computation-efficiency trade-off. In particular, a
higher sampling rate can decrease variance at the cost of increased
computation.  We show via simulations that the benefits of incredibly
high sampling rates is often negligible, as the contribution to the
variation is small in relation to to variation in the underlying
stochastic processes. Second, the subsampling procedure allows
inference to not depend on choice of dynamics for the health process.
Third, optimal estimating equations are derived that depend on the
instantaneous rate of the recurrent event process, which increase
increase efficiency of the subsampling procedure. Finally,
implementation can leverage existing software for functional data
analysis, leading to fast adoption by domain scientists. We show how
to implement methods via standard mixed effects software. 

\section{Related prior work}



\section{Recurrent event process and associated high frequency data}

Suppose $n$ subjects are independently sampled with observed event
times~$\bfT_{i} = \{ T_{i,1}, \ldots, T_{i,k_i}\}$ over some observation
window $[0, \tau_i]$ for each subject $i = 1,\cdots, n$.  Assume the
event times are ordered, i.e., $T_{i,j} < T_{i,j^\prime}$ for $j <
j^\prime$. The window length, $\tau_i$, is the censoring time and is
assumed independent of the event process. Let~$N_{i} (t)$ denote the
associated counting process of $\bfT_{i}$; that is, $N_i (t) =
\sum_{j=1}^{k_i} 1 [ T_{i,j} < t ]$.  
In this section, we assume a single-dimensional health process~$\bfx_i
= \{ x_i (s) \}_{0 < s < \tau_i}$ for each participant is measured at
a dense grid of time points.  Accelerometer, for example, is measured
at a rate of 32Hz (i.e., $32$ times per second). Electrodermal
activity (EDA), on the other hand, is measured at a rate of 4Hz (i.e.,
4 times per second).  Given the high frequency nature of the sensor 
measurements, this paper assumes the process is measured continuously.   

Let~$H_{i,t}^{NX} = H_{i,t}^{N} \otimes H_{i,t}^{X}$ be the $\sigma$-field
generated by all past values~$(N_i (s), x_i (s))_{0 \leq s \leq t}$.
The instantaneous risk of an event at time~$t$ is assumed to depend on
the health process, time-in-study, and the event history through a
fully parametric conditional hazard function:
\begin{equation}
\label{eq:hazard}
h_i \left( t \Given H_{i,t}^{NX} ; \theta \right) =
\lim_{\delta \to 0} \delta^{-1} \pr \left( N_i(t+\delta) - N_i(t)
  \given H_{i,t}^{NX}; \theta \right).
\end{equation}
where~$\theta$ is the parameter vector. Equation~\eqref{eq:hazard}
states the hazard may depend on past values of the event history and
covariates.  For high frequency physiological data, we assume that 
current risk is a linear functional of the health process over some
recent window of time; that is, 
\begin{equation}
\label{eq:hazardlinear}
h_i \left( t \given  H_{i,t}^{NX} ; \theta \right) = 
h_0 (t) \exp \left( g_t \left( H_{i,t}^{N} \right)^{\prime} \alpha
  + \int_{t-\Delta}^t x_i (s) \beta(s) ds  \right)
\end{equation}
where~$\Delta$ is an unknown window-length, and $g_t( H_{i,t}^N ) \in
\mathbb{R}^p$ is a $p$-length feature vector summarizing the
event-history and time-in-study information. The final
term~$\int_{t-\Delta}^t x_i(s) \beta(s) ds$ reflects the unknown
linear functional form of the impact of the time-varying covariate on
current risk.   

An alternative to~\eqref{eq:hazardlinear} would be to construct
features from the sensor data history~$f_t ( H_{i,t}^{X}) \in
\mathbb{R}^q$ and incorporated these features in the place of the
final term. Our current approach builds linear features of~$H_{i,t}^X$
directly from the integrated history, avoiding the feature
construction problem -- a highly nontrivial issue for high frequency
time-series data.  The main caveat is the additional
parameter~$\Delta$; however, we show that so long as the
estimated~$\hat \Delta$ exceeds~$\Delta$, then resulting estimation is
unbiased albeit at a loss of efficiency.  Moreover, sensitivity
analysis can be peformed to determine how choice of $\hat \Delta$
affects inference.  One limitation of the approach presented here is
that only fully parametric hazard models may be fit to the data.
However, a spline model for the log baseline hazard affords sufficient
model flexibility.

\subsection{Measurement-error models with event processes}

One potential criticism for~\eqref{eq:hazard} is that the health
process may be measured with error.  
A common mathematical strategy for joint models is to consider an
unobservable, latent process~$\eta_i$ such that~$\bfT_i \indep x_i
\given \eta_i$. Take~$\eta_i$ to be a zero-mean Gaussian process with 
\begin{equation}\label{eq:jm}
x_i(t) = \eta_i (t) + \epsilon_i (t),\quad \text{ and } \log h_i (t
\given \eta ) = \log h_0 (t) + g_t \left( H_{i,t}^N \right)^{\prime}
\alpha + \int_{t-\Delta}^t \eta_i (s) \beta (s) ds
\end{equation}
where~$\epsilon (t)$ is a white-noise measurement error. Thus,
$\eta_i (t)$ is the ``true and unobserved value of the longitudinal
outcome'' \citep[Sec. 2.1, pp.3]{Rizopoulos2010}. The conditional hazard
function given~$\bfx_i$ is given by
\[
\pr \left ( T_{i,j} = t \Given \H_{i,\tau_i}^X, \{ T_{i,j^\prime} < t \}_{j^\prime <
  j}, T_{i,j} \geq t \right) = E \left( h_i (t \given \eta) \exp
\left( \int_{T_{i,j-1}}^t h_i (\eta (s)) ds \right) \given \bfx_i
\right).
\]
Therefore, it appears that the conditional hazard function depends 
not only on past $x$-values, but also on future $x$-values.
Therefore, the processes~$\bfx$ and~$\bfT$ do not satisfy
\emph{independent evolution}~\citep{DempseyPMCC2}.
This is quite unnatural, as~\eqref{eq:jm} suggests current risk
depends on future values of the sensor process.
To alleviate this issue, we employ the strategy of taking~$\epsilon
(t) \equiv 0$ and treating~$\bfx_i$ as a mean-zero Gaussian process.
Note, the proposed inferential procedure will not rely on this
modeling assumption being correct; instead, we will use the random
subsampling mechanisms to address model misspecification.

\subsection{Likelihood calculation}  

For the sake of notational simplicity, we leave the dependency of the
conditional hazard function on~$H_{i,t}^{NX}$ implicit, and write~$h_i
(t; \theta)$. The component of the log-likelihood related to the event
process is then given by
\[
  L_n (\theta) = \sum_{i=1}^{n} \left ( \sum_{j=1}^{k_i} 
  \left[ \log \left( h_i \left( T_{i,j}; \theta
      \right) \right) - H_i \left( T_{i,j}, T_{i,j-1}; \theta \right) 
  \right] - H_{i} \left( \tau_i, T_{i,k_i}; \theta \right) \right) 
\]
where~$T_{i,0} = 0$, $T_{i, k_i + 1} = \tau_i$, and
\[
H_{i} (T_{i,j}, T_{i,j-1}; \theta) = \int_{T_{i,j-1}}^{T_{i,j}} h_{i} (t; \theta) dt.
\]
Solving the associated score equations~$U_n (\theta) = 0$ yields the
maximum likelihood estimator~$\hat \theta$, where
\[
U_n (\theta) = \sum_{i=1}^{n} \left ( \sum_{j=1}^{k_i} 
  \left[ \frac{h^{(1)}_i (T_{i,j}; \theta)}{h_i (T_{i,j}; \theta)} - H^{(1)}_i
    (T_{i,j}, T_{i,j-1}; \theta) \right]  - H^{(1)}_{i} (\tau_i,
  T_{i,k_i}; \theta) \right) 
\]
and~$h_i^{(1)} (T_{i,j}; \theta)$ and $H_i^{(1)} (T_{i,j}, T_{i,j-1}; \theta)$
are the derivatives of~$h_i^{(1)} (T_{i,j}; \theta)$ and~$H_i^{(1)}
(T_{i,j}, T_{i,j-1}; \theta)$ with respect to $\theta$.

In classical joint models~\citep{Henderson2000, Tsiatis2004},
time-varying covariates~$x_i (t)$ are observed only intermittently at
appointment times. In our current setting, maximizing the likelihood
is computationally prohibitive since for any~$\theta$ we must compute
the cumulative hazard functions~$H_{i} (T_{i,j}, T_{i,j-1}; \theta)$
which require integration of~$h_i(t;\theta)$ given
by~\eqref{eq:hazardlinear} which itself depends on the
integral~$\int_{t-\Delta}^t x_i (s) \beta(s) ds$.  
That is, the risk model now depends on an integrated past history of
the time-varying covariate which leads to severe increase in
computational complexity. 

\subsection{Probabilistic subsampling framework}

To avoid both the statistical modeling challenge and the computational
burden, we employ a point-process subsampling design to obtain
unbiased estimates of the cumulative hazards for each subject. The
subsampling procedure treats the collected sensor data as a set of
\emph{potential observations}. 
Suppose covariate information is sampled at times drawn from an
independent inhomogeneous Poisson point process with known
intensity~$\pi_i (t)$. At a subsampled time~$t$, the \emph{windowed
  covariate history} $\{ x_i (t-s)\}_{0 \leq s \leq \Delta}$ is
observed. Optimal choice of~$\pi_i (t)$ is beyond the scope of this
paper; however, the literature suggests setting proportional to the
hazard function~$h_i (t; \theta)$.

An estimator is design-unbiased if its expectation is equal to that
parameter under the probability distribution induced by the sampling
design~\citep{Cassel1977}. Let~$D_i \subset [0,t_i]$ denote the set of
subsampled points.  Note, by design, this set is distinct from the set
of event times, i.e.,~$\bfT_i \cap D_i = \emptyset$.  Under
subsampling via $\pi_i (t)$, then we may compute the Horvitz-Thompson
estimator 
\[
\hat H_{i} (T_{i,j}, T_{i,j-1}; \theta) = \sum_{t \in D_i \cap [T_{i,j-1},
  T_{i,j}]} \frac{ \lambda^{(1)} ( t \given \{ x(t - s) \}_{0 < s <
    \Delta}; \theta ) }{ \pi (s) }
\]
Let~$N = \{ T_{i,j} \}_{i =1, j=1}^{n,k_i}$ denote the complete set of
event times.  Then an alternative design-unbiased estimator of the
cumulative hazards is given by
\begin{equation}
\label{eq:WPest}
\hat H_{i} (T_{i,j}, T_{i,j-1}; \theta) = \sum_{t \in (N \cup D) \cap [T_{i,j-1},
  T_{i,j}]} \frac{ \lambda^{(1)} ( t \given \{ x(t - s) \}_{0 < s <
    \Delta}; \theta ) }{ \pi (s) + \lambda ( t \given \{ x(t - s) \}_{0 < s <
    \Delta}; \theta) }
\end{equation}
Equation~\eqref{eq:WPest} is the estimator suggested
by~\cite{Waagepetersen2008}.  This estimator depends on the
superposition of the event and subsampling processes.  This estimator
has been shown to be more efficient than the Horvitz-Thompson
estimator; therefore, we restrict our attention to~\eqref{eq:WPest}
for the remainder of this paper. Let
\[
w (t; \theta) = \frac{\pi (t)}{\pi (t) + \lambda (t \given \{
  x(t - s) \}_{0 < s < \Delta}; \theta)}
\]
Then the resulting estimating equations are given by
\begin{equation}
\label{eq:approxscore}
\hat{U}_n = \sum_{t \in N} w(t; \theta) \frac{\lambda^{(1)} (t \given \{
  x(t - s) \}_{0 < s < \Delta}; \theta)}{ \lambda ( t \given \{
  x(t - s) \}_{0 < s < \Delta}; \theta)}  - 
\sum_{t \in D} w(t; \theta) \frac{\lambda^{(1)} (t \given \{
  x(t - s) \}_{0 < s < \Delta}; \theta)}{ \pi (t) }
\end{equation}
Equation~\eqref{eq:approxscore} are approximate score functions built
via plug-in of the design-unbiased estimator of the cumulative hazard.

We note that this approach avoids modeling the sensor stream process,
which would be required if one were to consider a standard joint model
approach.  Sensor streams such as accelerometer and skin-conductance
relate to complex dynamical systems.  The subsampling
approach avoids this difficult modeling task entirely.


\section{Longitudinal functional principal components within
  event-history analysis}

Probabilistic subsampling converts the single sensor stream~$\bar x_i$
into a sequence of functions observed repeatedly at sampled
times~$D_i$ and event times~$N_i$. Such a data structure is commonly
referred to as \emph{longitudinal functional data}~\ref{Xiao2013,
  GoldSmith2015}.
Given the large increase in longitudinal functional data in recent
years, corresponding analysis has received much recent
attention~\citep{Morris2003, MorrisCarroll2006,
  Baladandayuthapani2008, Di2009, Greven2010, Staicu2010,
  ChenMuller2012, LiGuan2014}.
In this paper, we combine work by~\cite{Park2018}
and~\cite{Goldsmith2011} to construct a computationally efficient
penalized functional method for solving the estimation equations~$\hat
U_n (\theta)$.

In our setting, probabilistic subsampling leads to observation
times~$t \in N \cup D$ at which 


The marginal covariance~\cite{Park2018} allows a single 
\begin{itemize}
\item {\bf Step 1}: Estimate~$\hat \mu(s,t)$ via sandwich smoother for
  all~$s \in [0,\Delta]$ and~$t \in \cup_{i=1}^n \left( D_i \cup N_i
  \right)$. Define~$\tilde X (s,t) = X(s,t) - \hat \mu(s,t)$ to be a
  mean-zero process for each~$t$.
\item {\bf Step 2}: Compute the pooled sample covariance
\[
\hat \Xi (s , s^\prime) = 
\left( \sum_{i=1}^n \left | N_i \cup D_i \right |\right)^{-1}
\left( \sum_{i=1}^n \sum_{t \in N_i \cup D_i} \tilde X (s,t) \tilde X
  (s^\prime, t) \right).
\]
Compute at a fine set of grid points. This is an estimate of 
\[
\Xi (s, s^\prime) = \int_{\mathcal{T}} c( (s,T), (s^\prime, T) ) g(T) dT
\]
where~$g(T)$ is a superposition of the event process and the
subsampling process.
\item {\bf Step 3}: Take spectral decomposition~$\{ \hat \phi_k (s),
  \hat \lambda_k \}_{k \geq 1}$. Choose~$K_x \geq K_b$ large enough.
\item {\bf Step 4}: Compute
\[
\hat c_{i,k} (t) = \int_{t-\Delta}^t \tilde X_i (s,t) \hat \phi_k (s) ds
\]
Then
\[
X(s,t) \approx \hat \mu(s,t) + \sum_{k=1}^{K_x} \hat c_{i,k} (t)
\phi_k (s).
\]
\item {\bf Step 5}: Plugging both the spline and KL decomposition we
  have
\begin{align*}
\int_{t-\Delta}^t X(s,t) \beta(s) ds 
  &= \int_{t-\Delta}^t \left[ \hat \mu(s,t) + \sum_{k=1}^{K_x}
    c_{i,t,k} \phi_k (s) \right] \times \left[ \sum_{k^\prime=1}^{K_b}
    \beta_k \psi_k (s) \right] \\
  &= [ M_{i,t}^\prime + C_{i,t}^\prime J_{\phi, \psi} ] \beta
\end{align*}
where~$M_{i,t}$ is a vector with~$i$th entry equal
to~$\int_{t-\Delta}^t \hat \mu(s,t) \psi_k (s) ds$, $C_{i,t} =
(c_{i,t,1}, \ldots, c_{i,t,K_x}) \in \mathbb{R}^{K_x \times 1}$,
$\beta = (\beta_1, \ldots, \beta_{K_b}) \in \mathbb{R}^{K_b \times 1}$
and~$J_{\phi, \psi} \in \mathbb{R}^{K_x \times K_b}$ where the~$(i,j)$
entry is given by
\[
\int_{t-\Delta}^t \phi_{i} (s) \psi_{j} (s) ds.
\]
\end{itemize}

We re-write the hazard function as:
\[
h_i (t; \theta) = \exp \left( Z_{i,t}^\prime, \alpha + [M_{i,t} + C_{i,t}^\prime
  J_{\phi, \psi} ] \beta \right) = \exp \left( W_{i,t}^\prime \theta \right),
\]
where~$W_{i,t}^\prime = [ Z_{i,t}^\prime (M_{i,t}^\prime +
C_{i,t}^\prime J_{\phi, \psi}) ]$ and~$\theta = (\alpha,
\beta)$. Then, the score function is:
\[
\hat U_n (\theta) = \sum_{i=1}^n \left \{  \sum_{t \in N_i}
  \frac{h_i^{(1)} (t; \theta)}{h_i (t; \theta)} - \hat H_i^{(1)}
  (\tau_i, 0; \theta) 
\right \}
\]
where
\[
\hat H_i^{(1)} (\tau_i; \theta) = \sum_{u \in N_i \cup D_i} \frac{
  h^{(1)}_i (u; \theta)}{\pi_i (u) + h_i (u; \theta)}.
\]
Note that
\[
\frac{d}{d\theta} \log h_i (t; \theta) = \frac{h^{(1)}_i (t;
  \theta)}{h_i (t; \theta)}.
\]
Therefore, these terms have very simple forms!!! 
\[
\frac{d}{d\theta} \log h_i (t; \theta) = W_{i,t}
\]
On the other hand,
\[
\frac{d}{d \theta} \log \left( \pi_i (t) + h_i (t;\theta) \right) = 
\frac{\exp \left( W_{i,t}^\prime \theta \right)}{\pi_i (t) + \exp
  \left( W_{i,t}^\prime \theta \right)} W_{i,t}
\]
Therefore,
\begin{align*}
\hat U_n (\theta) 
  &= \sum_{i=1}^n \left \{ \sum_{t \in N_i} W_{i,t} -
    \sum_{u \in N_i \cup D_i} \frac{\exp \left( W_{i,u}^\prime \theta
    \right)}{\pi_i (t) + \exp \left( W_{i,u}^\prime \theta \right)}
    W_{i,u} \right \} \\
  &= \sum_{i=1}^n \left \{ \sum_{t \in N_i} \frac{\pi_i (t)}{\pi_i (t)
    + \exp \left( W_{i,u}^\prime \theta \right)} W_{i,t} -
    \sum_{u \in D_i} \frac{\exp \left( W_{i,u}^\prime \theta
    \right)}{\pi_i (t) + \exp \left( W_{i,u}^\prime \theta \right)}
    W_{i,u} \right \} \\
  &= \sum_{i=1}^n \left \{ \sum_{t \in N_i} w_{i} (t; \theta) W_{i,t} -
    \sum_{u \in D_i} (1 - w_i (u; \theta)) W_{i,u} \right \} \\
\end{align*}


\section{Prediction}
We wish to estimate the risk at non-sampled points.

For prediction, we use sparse FPCA (Yao 2005) 
\begin{itemize}
\item Take $\hat c_{k} (s,t)$ for~$t \in N \cup D$ and~$s \in
  [0,\Delta]$
\item 
\end{itemize}


Take $\{ X (s) \}_{0 < s < \Delta}$. Let~$\psi_{j} (s)$ is an
orthonormal basis for $L^2 [ 0,1 ]$.  Then

\[
X(s) - \mu(s) 
\]

\[
- b^\prime D b
\]


\section{Theoretical analysis}

\begin{assumption}[Event process assumptions]

\end{assumption}

\begin{assumption}[Covariance assumptions]

\end{assumption}

\begin{assumption}[Functional data assumptions]

\end{assumption}

\begin{thm}[No reconstruction error]

\end{thm}

\begin{thm}[Reconstruction error but~$\hat \Delta > \Delta$]
\label{eq:biaseddown}
\end{thm}

Theorem~\ref{eq:}



\begin{thm}[When~$\hat \Delta < \Delta$] 

\end{thm}



\section{Simulations}

\subsection{Unbiased estimation}

\subsection{Unbiased estimation}

\section{Case Study: }

\bibliographystyle{plainnat}
\bibliography{si-fda-refs}

\end{document}