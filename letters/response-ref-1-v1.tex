%%%%%%%%%%%%%%%%%%%%%%%%%%%%%%%%%%%%%%%%%
% Plain Cover Letter
% LaTeX Template
%
% This template has been downloaded from:
% http://www.latextemplates.com
%
% Original author:
% Rensselaer Polytechnic Institute (http://www.rpi.edu/dept/arc/training/latex/resumes/)
%
%%%%%%%%%%%%%%%%%%%%%%%%%%%%%%%%%%%%%%%%%

%----------------------------------------------------------------------------------------
%	PACKAGES AND OTHER DOCUMENT CONFIGURATIONS
%----------------------------------------------------------------------------------------

\documentclass[11pt]{letter} % Default font size of the document, change to 10pt to fit more text

\usepackage{newcent} % Default font is the New Century Schoolbook PostScript font
%\usepackage{helvet} % Uncomment this (while commenting the above line) to use the Helvetica font

% Margins
\topmargin=-1in % Moves the top of the document 1 inch above the default
\textheight=8.5in % Total height of the text on the page before text goes on to the next page, this can be increased in a longer letter
\oddsidemargin=-10pt % Position of the left margin, can be negative or positive if you want more or less room
\textwidth=6.5in % Total width of the text, increase this if the left margin was decreased and vice-versa

\let\raggedleft\raggedright % Pushes the date (at the top) to the left, comment this line to have the date on the right
\usepackage{amsthm, amsfonts}
\usepackage{xcolor}
\usepackage{hyperref}


\def\Y{{\bf Y}}

\begin{document}

%----------------------------------------------------------------------------------------
%	ADDRESSEE SECTION
%----------------------------------------------------------------------------------------

\begin{letter}{Professor
	Peter Glynn\\
	Editor, {\em Journal of Applied Probability}}

%----------------------------------------------------------------------------------------
%	YOUR NAME & ADDRESS SECTION
%----------------------------------------------------------------------------------------

%\begin{center}
%\large
%\end{center}
%\vfill

\signature{Walter Dempsey\\
University of Michigan\\
Department of Biostatistics\\
1415 Washington Heights\\
Ann Arbor, MI 48103} % Your name for the signature at the bottom

%----------------------------------------------------------------------------------------
%	LETTER CONTENT SECTION
%----------------------------------------------------------------------------------------

\vspace{5mm}

\newpage

{\bf Response to Referees}

We appreciate all of the very helpful comments from the editor, associate editor, and the two referees which has helped improve the paper substantially. We have addressed the concerns as best as possible in this revision. We provide a further point-by-point explanation below. Any comments in {\it italics} indicate a statement from the referee report.

{\bf Response to Associate Editor}
\begin{enumerate}
\item {\it This paper was reviewed by two referees, who tend to agree in their positive assessments of the value and novelty of your work, and provide some very specific comments for further improvements.}

\vspace{5mm}
We thank the associate editor.  We have addressed the two reviewers' specific comments.  Responses to their concerns are detailed below and have significantly improved the clarify of this manuscript.
\vspace{5mm}

\item {\it Expanded literature review on existing modeling approaches. Both referees raise this point, and I agree that the contrast with existing works could be more explicit. The second referee provides some specific suggestions on relevant works.}

\vspace{5mm}
XXXX
\vspace{5mm}

\item {\it There appear to be some missing definitions of notation, and some transitions/equations need more details to justify them. The latter could potentially be put in the supplement to save space. The former should be added to the main manuscript to improve clarity.}

\vspace{5mm}
XXXX
\vspace{5mm}

\item {\it As the first reviewer, I am confused by $\varepsilon(t) = 0$ on page 4, line 39, in line with equation (3) and measurement error methodology. I suspect comment 7 from the 2nd reviewer alludes to the same thing. I suspect this may be a notation issue rather than the modeling issue, but either way, it needs to be clarified.}

\vspace{5mm}
XXXX
\vspace{5mm}

\item {\it Practical considerations, such as window size and smoothing parameter selection, need clarification.}

\vspace{5mm}
XXXX
\vspace{5mm}

\item {\it Reviewer 2 suggests adding additional details on proof steps and lemma clarifications. I can't entirely agree that the proofs should go in the main manuscript as the paper is already on the longer side. When making adjustments to this section (theoretical analysis), please keep at most its current length, and move things to supplement as needed.}

\vspace{5mm}
XXXX
\vspace{5mm}

\item {\it Both reviewers suggest additional simulation studies. I agree that it would be good to have a numerical benchmark of the magnitude of the computational issue that you allude to in lines 16-28 on page 5, especially for JCGS readership. I don’t think you need references for this, as the reviewer asks, as a numerical example would be more helpful and convincing in this case. However, I think you need another method for comparison, as the simulations are only done for your approach with different sampling rates. Perhaps some suggestions from the first referee could be useful. If some of these alternatives are computationally prohibitive, then doing a computational numerical comparison on even 1 rep will provide a convincing argument for why those methods are not tried.}

\vspace{5mm}
XXXX
\vspace{5mm}

\item {\it I also have additional comments that were not raised. A major one is the absence of code. A user-friendly software with the method’s implementation and documentation is a must for accepted JCGS papers. Please provide the code so that the results are reproducible.}

\vspace{5mm}
XXXX
\vspace{5mm}

\item {\it
A minor one is the absence of truncation $K_x$ in the equation on line 37, page 8. Please reread everything carefully to ensure all notations are defined and the equations are consistent with the text.}

\vspace{5mm}
XXXX
\vspace{5mm}

\item {\it Another minor one is your choice of smoothing the entire covariance, including the diagonals (page 8, lines 13-18). It’s unclear to me what the rationale is for changing this step compared to Park and Staicu, 2015, and how much it matters for the final performance.}

\vspace{5mm}
XXXX
\vspace{5mm}

\item {\it Expanded literature review on existing modeling approaches. Both referees raise this point, and I agree that the contrast with existing works could be more explicit. The second referee provides some specific suggestions on relevant works.}

\vspace{5mm}
XXXX
\vspace{5mm}

\end{enumerate}

\newpage
{\bf Response to Reviewer 1}
\vspace{5mm}

I am very grateful to the referee for the incredibly detailed review.  The detailed suggestions and comments helped improve the paper in numerous ways.  See below for our responses to each particular reviewer comment.

\begin{enumerate}
\item {\it The introduction it quite lacking. Please add more background literature review of prevailing/latest approaches (particularly in the area of joint modeling of longitudinal and time to event data) to the introduction.}

\vspace{5mm}
XXXXX
\vspace{5mm}

\item {\it The measurement error part can be added to the end with other exten-
sions. It does not seem to play a major role in developing the algorithm.}

\vspace{5mm}
We agree.  This point was over-emphasized in the original manuscript and so has been moved to the
\vspace{5mm}

\item {\it In equationn (4), page 4, do you mean that equation (2) and the mesurement error part $x_i (t) = \eta_i (t) + \epsilon_i (t)$ imply conditional hazard equation in (3)?  If yes, then why is $\int_{t-\Delta}^t \epsilon_i (s) \beta(s) ds = 0$?  Or are you assuming a model?  Please add more clarity.}

\vspace{5mm}
XXXXX
\vspace{5mm}

\item {\it Several notations in section 2.1 such as ${\bf x}_i, {\bf H}_{i,\tau_i}^X$ are not defined.}

\vspace{5mm}
XXXXX
\vspace{5mm}

\item {\it How did you get the equation in line 24, page 4.}

\vspace{5mm}
XXXXX
\vspace{5mm}

\item {\it In line 39, page 4, by assuming $\epsilon(t) = 0$, you assume there is no measure- ment error? Why is section 2.1 called measurement error if the error is 0? Should this be named missing data section instead?}

\vspace{5mm}
The key point raised on page 4 (no Section XX), is that if $\epsilon(t)$ is not equivalently zero, your model assumes risk of an event at time $t$ depends on both the current and future measurements of the process ${\bf x}_i$. We argue this is not sensible since current risk should not depend on future risk and thus set $\epsilon(t) = 0$.
\vspace{5mm}

\item  {\it Is $\theta$ defined anywhere prior to section 2.2?}

\vspace{5mm}
It was used in equation (1) to refer to all the parameters underlying the model to distinguish it from the functional parameter~$\beta (s)$.
We do XXXX to make this clearer.
\vspace{5mm}

\item {\it On page 5, lines 16 - 28 you claim that computing the double integral is computationally expensive. Add some references. Also, to give a greater insight to the extent of the computational cost, solve the score equation in lines 8,9 i.e. without the sub-sampling framework in the simulations (there is another comment on this for the simulation section). Reporting a rough summary of computational cost such as ” it took x min to solve equations involving n samples” here, would help the reader understand
the extent of the problem.}

\vspace{5mm}
We explain that if we were to compute it would have to be as a function of $\beta(s)$.  We implement a naive and demonstrate numerically the computational cost.  Specifically, each computation in simulation takes XXseconds, and we must compute this XX times to compute the likel
\vspace{5mm}

\item {\it Page 5, line 51, what is $t_i$?}

\vspace{5mm}
This should be $\tau_i$ which is the censoring time of the observation process for individual~$i$.
\vspace{5mm}

\item {\it Check the notation in equation (6) page 6. Some subscripts seem to be missing. Add some details on how to obtain this from the original score
equation. What is the approximation here?}

\vspace{5mm}
We have included correct subscripts in the revision.  Derivation of (6) follows from plugging in (4) into equation the score equations.  We now make this clear and provide a derivation in the supplementary materials for completeness.
\vspace{5mm}

\item{\it The remark 2.1 can be added the end of section 2.2. This along with
comment 8 will naturally lead the way your proposed method.}

\vspace{5mm}
We thank the referee for this suggestion.  We have now used Remark 2.1 and comment 8 on the computational costs to the end of section 2.2 to motivated our proposed method.
\vspace{5mm}

\item {\it Page 7, line 48, in the definition of the marginal covariance, it should be $dT$ in the integral.}

\vspace{5mm}
That is correct.  We have fixed the definition as suggested.
\vspace{5mm}

\item {\it Page 7, line 48, in the definition of the marginal covariance, what is $\tau$?}

\vspace{5mm}
The term $\tau$ refers to the censoring time of the observation process. We make this clear in Section XXX, page XX in the revision.
\vspace{5mm}


\item {\it s there a difference between the mean and covariance functions for $y = 0$ and $y = 1$ especially if points in $D_i$ and $T_i$ are spread roughly evenly on a grid even if the sets are disjoint? For example, pick a grid of equidistant points and number them from $1-10$. If all odd points are events in $T_i$ and even ones are sampled in $D_i$, won't the pooled sample covariance for both be the same}

\vspace{5mm}
The event times~$T_i$ depend heavily on the event intensity function~$h_i (t;\theta)$ and will not be guaranteed to be uniformly spread across points in $[0,\tau]$.  The subsampling times~$D_i$ depend on the subsampling intensity function~$\pi_i(t)$ and
\vspace{5mm}

\item {\it Add a plot of the mean, covariance functions for $y=0$ and $y=1$.}

\vspace{5mm}
We have added mean and covariance functions to the supplementary materials as suggested.  The plots demonstrate distinct behavior in the means for $y=0$ and $y=1$; while the covariance functions are XXXX.
\vspace{5mm}

\item {\it Page 9, line 6, isn’t the integral in (2) $\int_{t-\Delta}^t X(s) \beta(s)ds$? The whole development assumes the integral to be $\int_{t-\Delta}^t X(t−s) \beta(s)ds$. The interpretation of the coefficient function $\beta()$ would be different for two cases. Please clarify.}

\vspace{5mm}
We apologize for the confusion.  In Section 3.1, we started by defining the double-indexed $X(t,s)$ to be equal to $X(t-s)$ in order to make the notation a bit cleaner but we see that this has added confusion.  We not add a remark to make this more clear.  The reason for the notation was that writing $X(t,s)$ allows us to write the integrals over $0$ to $\Delta$ rather than having to constantly index integrals by $t$.  Moreover, we thought it helps the reader to understand we are saying ``the process $s$ units prior to time~$t$.''
\vspace{5mm}

\item {\it Page 9, equation 7, please define all of the notations before using them.}

\vspace{5mm}
We have re-written this paragraph to first define the notation and then present equation (7).
\vspace{5mm}

\item {\it Page 12, line 48, what is proposition 3.6? You mean Lemma?}

\vspace{5mm}
That is correct, we have fixed this issue.
\vspace{5mm}

\item {\it When are the assumptions satisfied or not satisfied? An example of a
situation where the functional and event process assumptions are easily satisfied will be useful.}

\vspace{5mm}
We have shown that the functional process satisfies these moment conditions under boundedenss which is
\vspace{5mm}

\item {\it All the modes of convergences in Lemma 3.5 and its proofs should be made clear.}

\vspace{5mm}
XXXxXX
\vspace{5mm}

\item {\it Outline the proof of Lemma 3.5 first and then prove the details.  It will be easier to follow.  Alternatively, we }

\vspace{5mm}
We have added a sketch of the proof with an outline as part of the supplementary details.
\vspace{5mm}

\item {\it In Lemma 3.6 state the theorem clearly.  What condition do yu show in the proof that establishes optimality}

\vspace{5mm}
We have now included the statement of how we define optimality in the main paper to make the theorem statement clear.  Proof is kept in the supplementary materials for conciseness.
\vspace{5mm}

\item {\it Combine section 3.6 and 3.7 in to a single result that establishes a confidence band for $\beta (t)$. Also, wont Lemma 3.5 lead to a confidence band for $\hat \beta (t)$? Are the two bands same?}

\vspace{5mm}
We discuss the penalization due to the potential complexity of the.  The confidence bands are based on Lemma 3.5.  We make this clear in the revision.  The issues are still thre and we keep the discussion.  We keep as 2 separate sections since they cover different issues.
\vspace{5mm}

\item The ``original'' score equation Page 5, line 9 involves
$$
H_i (\tau_i ;\theta) = \int_{t=0}^{\tau_i} h_i (t; \theta)dt = \int_0^{\tau_i} \left[ h_0 (t;\gamma) \exp \left( g_t (H_{i,t}^N)^\top \alpha + \int_{s=t-\Delta}^t x_i (s) \beta(t) ds \right) \right] dt
$$
In the subsampled version you suggest in equation 6, page 6, you essentially replace $H_i (\tau_i; \theta)$ computation with $h_i (u; \theta)$, $u \in D_i$. How much time would it take to solve the full likelihood equation that will involve the above equation (1) directly? With the basis expansion strategy, you
would will be estimating the same number of parameters. This type of approach should be added to simulations and evaluated for accuracy and time complexity.

\vspace{5mm}
XXXXX
\vspace{5mm}

\item {\it How about using a concurrent model for $h_i (t;\theta) = h_0 (t; \theta) \exp (g_t (H_{i,t}^N)^\top \alpha + x_i (t) \beta (t))$. This does not look at the past window like the suggested model but this drawback might be compensated with the constructed features $g_t (H_{i,t}^N)$ that capture the past. The concurrent approach might be much less expensive computationally. This can be added as an alternative to the simulations.}

\vspace{5mm}
XXXXX
There is no obvious way to compute bias since the model will be incorrectly specified.  We do show the computation trade-off and that computing the cumulative hazard is still computationally demanding; therefore, after sampling non-event times, the added computation of the interior integral is not very expensive and avoids model misspecification.
\vspace{5mm}

\item {\it In section 5.2.1 what is uncongeniality? Does it have a mathematical
meaning?}

\vspace{5mm}
uncongenial comes from Meng. We cite and provide a short formal definition with some intuition.
\vspace{5mm}

\item Page 23, Figure 2. EDA seems to spike at the end. Even AI seems to be
turning up. Add longer history to both the Figures.


\item {\it Doesn’t Figure 3 suggest smaller ∆?}

\vspace{5mm}
We agree.  We re-run the analysis with a window length of 15 minutes as well and move the current analysis as a sensitivity to the supplementary materials.
\vspace{5mm}

\item {\it What features were constructed for the real data analysis?}

\vspace{5mm}
As this was illustrative we only included time-of-day.
\vspace{5mm}


\end{enumerate}

\newpage

{\bf Response to Reviewer \#2}

I am incredibly grateful for this review.  I agree that the paper has evolved beyond the current title.  This was not updated in the prior submission simply to ensure continuity of presentation (i.e., didn't want to change from first submission).  Here we replace with the title ``Addressing selection bias and measurement error in COVID-19 case count data using auxiliary information''.  We next address the reviewers specific comments.

\begin{enumerate}
\item {\it The paper states some facts about the SARS-CoV-2 virus that are outdated, for example “Upon infection with SARS-CoV-2, an incubation period starts and lasts approximately five days [Lauer et al., 2020] before the viral load is high enough to be detected.” (Page 5). Lauer was based on the original
strain, we see different disease dynamics with the delta strain. Perhaps to be more precise, the author could clarify that this is an example of the original SARS-CoV-2 incubation rather than definitative as there will likely be future variants that will change this further. Similarly, on page 13, the serial
interval of 7 is stated as fact – this is again based on the original strain rather than the current disease dynamics. Stating these numbers as an example rather than definitive would be preferable, I believe.}
\vspace{5mm}
\end{enumerate}


\end{letter}






\end{document}





